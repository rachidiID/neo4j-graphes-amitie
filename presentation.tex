\documentclass[aspectratio=169]{beamer}

% Thème
\usetheme{Madrid}
\usecolortheme{seahorse}

% Packages
\usepackage[utf8]{inputenc}
\usepackage[T1]{fontenc}
\usepackage[french]{babel}
\usepackage{graphicx}
\usepackage{listings}
\usepackage{xcolor}
\usepackage{tikz}
\usepackage{booktabs}

% Couleurs Neo4j
\definecolor{neo4jblue}{RGB}{0,136,206}
\definecolor{neo4jdark}{RGB}{0,90,140}
\definecolor{codegreen}{RGB}{0,128,0}

\setbeamercolor{structure}{fg=neo4jblue}
\setbeamercolor{palette primary}{bg=neo4jblue,fg=white}
\setbeamercolor{palette secondary}{bg=neo4jdark,fg=white}

% Configuration Cypher
\lstdefinelanguage{Cypher}{
  keywords={MATCH, CREATE, RETURN, WHERE, WITH, ORDER, BY, LIMIT},
  keywordstyle=\color{neo4jblue}\bfseries,
  commentstyle=\color{codegreen},
  stringstyle=\color{red},
  basicstyle=\ttfamily\small,
  breaklines=true,
  frame=single
}

% Informations
\title{Graphe d'amitié entre étudiants}
\subtitle{Projet Neo4j - Base de données orientée graphes}
\author{Rachidi et équipe}
\institute{Université}
\date{24 novembre 2025}

\begin{document}

% Page de titre
\begin{frame}
  \titlepage
\end{frame}

% Sommaire
\begin{frame}{Sommaire}
  \tableofcontents
\end{frame}

% Section 1
\section{Introduction}

\begin{frame}{Contexte}
  \begin{columns}
    \begin{column}{0.5\textwidth}
      \textbf{Problématique}
      \begin{itemize}
        \item Modéliser un réseau social
        \item Analyser les relations d'amitié
        \item Recommander de nouveaux amis
        \item Visualiser les communautés
      \end{itemize}
    \end{column}
    
    \begin{column}{0.5\textwidth}
      \textbf{Pourquoi Neo4j ?}
      \begin{itemize}
        \item Base orientée graphes
        \item Relations = citoyens de 1ère classe
        \item Requêtes intuitives (Cypher)
        \item Performance sur les graphes
      \end{itemize}
    \end{column}
  \end{columns}
  
  \vspace{1cm}
  \begin{alertblock}{Objectif}
    Créer un graphe social complet avec analyses et recommandations
  \end{alertblock}
\end{frame}

\begin{frame}{Technologies utilisées}
  \begin{center}
  \begin{tikzpicture}
    % Neo4j
    \node[draw, rounded corners, fill=neo4jblue!20, minimum width=3cm, minimum height=1cm] at (0,2) {\textbf{Neo4j 5.x}};
    \node[below] at (0,1.5) {Base de données};
    
    % Cypher
    \node[draw, rounded corners, fill=green!20, minimum width=3cm, minimum height=1cm] at (4,2) {\textbf{Cypher}};
    \node[below] at (4,1.5) {Langage de requête};
    
    % Python
    \node[draw, rounded corners, fill=yellow!20, minimum width=3cm, minimum height=1cm] at (0,0) {\textbf{Python 3.x}};
    \node[below] at (0,-0.5) {Scripts d'analyse};
    
    % NetworkX
    \node[draw, rounded corners, fill=orange!20, minimum width=3cm, minimum height=1cm] at (4,0) {\textbf{NetworkX}};
    \node[below] at (4,-0.5) {Visualisation};
  \end{tikzpicture}
  \end{center}
\end{frame}

% Section 2
\section{Modélisation}

\begin{frame}{Modèle de données}
  \begin{center}
  \begin{tikzpicture}[scale=0.9]
    % Nœuds
    \node[circle, draw, fill=neo4jblue!30, minimum size=2cm] (etudiant) at (0,0) {\textbf{Etudiant}};
    \node[circle, draw, fill=green!30, minimum size=1.5cm] (cours) at (5,1) {\textbf{Cours}};
    \node[circle, draw, fill=orange!30, minimum size=1.5cm] (ville) at (5,-1) {\textbf{Ville}};
    
    % Relations
    \draw[->, thick, bend left=30] (etudiant) to node[above] {\small AMI\_AVEC} (etudiant);
    \draw[->, thick] (etudiant) to node[above] {\small ÉTUDIE} (cours);
    \draw[->, thick] (etudiant) to node[below] {\small VIT\_À} (ville);
  \end{tikzpicture}
  \end{center}
  
  \vspace{0.5cm}
  
  \begin{block}{Propriétés principales}
    \begin{itemize}
      \item \textbf{Etudiant} : nom, prénom, age, email, filière, hobbies
      \item \textbf{AMI\_AVEC} : depuis, force (1-10), type (proche/etudes/sport)
      \item \textbf{ÉTUDIE} : année, note, présence
    \end{itemize}
  \end{block}
\end{frame}

\begin{frame}[fragile]{Contraintes et index}
  \begin{lstlisting}[language=Cypher]
// Contraintes d'unicité
CREATE CONSTRAINT student_id_unique 
FOR (e:Etudiant) REQUIRE e.student_id IS UNIQUE;

CREATE CONSTRAINT cours_code_unique 
FOR (c:Cours) REQUIRE c.cours_code IS UNIQUE;

// Index pour performances
CREATE INDEX etudiant_nom_index 
FOR (e:Etudiant) ON (e.nom);

CREATE INDEX cours_code_index 
FOR (c:Cours) ON (c.cours_code);
  \end{lstlisting}
\end{frame}

% Section 3
\section{Implémentation}

\begin{frame}[fragile]{Création d'un étudiant}
  \begin{lstlisting}[language=Cypher]
CREATE (rachidi:Etudiant {
  student_id: 'ETU001',
  nom: 'Diallo',
  prenom: 'Rachidi',
  age: 22,
  email: 'rachidi.diallo@univ.fr',
  filiere: 'Informatique',
  niveau: 'M1',
  ville: 'Paris',
  hobbies: ['programmation', 'football'],
  date_inscription: date('2023-09-01')
})
  \end{lstlisting}
\end{frame}

\begin{frame}[fragile]{Relations d'amitié bidirectionnelles}
  \begin{lstlisting}[language=Cypher]
MATCH (rachidi:Etudiant {student_id: 'ETU001'})
MATCH (marie:Etudiant {student_id: 'ETU002'})

// Créer les deux sens
CREATE (rachidi)-[:AMI_AVEC {
  depuis: date('2023-09-15'),
  force: 9,
  type: 'proche'
}]->(marie)

CREATE (marie)-[:AMI_AVEC {
  depuis: date('2023-09-15'),
  force: 9,
  type: 'proche'
}]->(rachidi)
  \end{lstlisting}
\end{frame}

\begin{frame}{Jeu de données initial}
  \begin{center}
  \begin{tabular}{lccc}
    \toprule
    \textbf{Élément} & \textbf{Nombre} & \textbf{Détails} \\
    \midrule
    Étudiants & 6 & 3 villes différentes \\
    Cours & 5 & INFO301 à INFO305 \\
    Amitiés & 8 paires & 16 relations (bidirect.) \\
    Inscriptions & 17 & Avec notes et présence \\
    \bottomrule
  \end{tabular}
  \end{center}
  
  \vspace{0.5cm}
  
  \begin{exampleblock}{Diversité}
    Différentes filières, niveaux, hobbies, forces d'amitié
  \end{exampleblock}
\end{frame}

% Section 4
\section{Analyses}

\begin{frame}[fragile]{Requêtes de base}
  \textbf{1. Lister tous les étudiants}
  \begin{lstlisting}[language=Cypher]
MATCH (e:Etudiant)
RETURN e.nom, e.prenom, e.ville
ORDER BY e.nom;
  \end{lstlisting}
  
  \vspace{0.3cm}
  
  \textbf{2. Trouver les amis de Rachidi}
  \begin{lstlisting}[language=Cypher]
MATCH (e:Etudiant {nom: 'Diallo'})-[:AMI_AVEC]->(ami)
RETURN ami.nom + ' ' + ami.prenom as ami,
       ami.ville;
  \end{lstlisting}
\end{frame}

\begin{frame}[fragile]{Analyses sociales}
  \textbf{Étudiants les plus populaires}
  \begin{lstlisting}[language=Cypher]
MATCH (e:Etudiant)-[:AMI_AVEC]->(:Etudiant)
WITH e, COUNT(*) as nb_amis
RETURN e.nom + ' ' + e.prenom as etudiant,
       nb_amis
ORDER BY nb_amis DESC
LIMIT 5;
  \end{lstlisting}
  
  \vspace{0.3cm}
  
  \begin{block}{Résultats}
    \begin{itemize}
      \item Rachidi Diallo : 4 amis
      \item Marie Dupont : 4 amis
      \item Ahmed Ben Ali : 3 amis
    \end{itemize}
  \end{block}
\end{frame}

\begin{frame}[fragile]{Recommandations d'amitié}
  \begin{lstlisting}[language=Cypher]
MATCH (e1:Etudiant {nom: 'Diallo'})
      -[:AMI_AVEC]->()-[:AMI_AVEC]->(e2:Etudiant)
WHERE NOT (e1)-[:AMI_AVEC]->(e2) AND e1 <> e2
WITH e2, COUNT(*) as amis_communs
RETURN e2.nom + ' ' + e2.prenom as suggestion,
       amis_communs
ORDER BY amis_communs DESC
LIMIT 5;
  \end{lstlisting}
  
  \begin{alertblock}{Principe}
    Recommander des personnes avec qui on a des amis en commun
  \end{alertblock}
\end{frame}

\begin{frame}[fragile]{Performance académique}
  \begin{lstlisting}[language=Cypher]
MATCH (e:Etudiant)-[r:ÉTUDIE]->(c:Cours)
WITH e, AVG(r.note) as moyenne
RETURN e.nom + ' ' + e.prenom as etudiant,
       ROUND(moyenne * 100) / 100 as moyenne
ORDER BY moyenne DESC;
  \end{lstlisting}
  
  \vspace{0.5cm}
  
  \begin{center}
  \begin{tikzpicture}
    \begin{axis}[
      ybar,
      symbolic x coords={Rachidi, Marie, Ahmed, Sophie, Thomas, Laura},
      xtick=data,
      ymin=0, ymax=20,
      ylabel={Moyenne},
      xlabel={Étudiant},
      width=10cm, height=5cm
    ]
    \addplot coordinates {(Rachidi,14.5) (Marie,13.8) (Ahmed,12.3) 
                          (Sophie,15.2) (Thomas,11.7) (Laura,16.1)};
    \end{axis}
  \end{tikzpicture}
  \end{center}
\end{frame}

% Section 5
\section{Scripts Python}

\begin{frame}[fragile]{Connexion Neo4j}
  \begin{lstlisting}[language=Python]
from neo4j import GraphDatabase

class Neo4jConnection:
    def __init__(self, uri, user, password):
        self.driver = GraphDatabase.driver(uri, 
                                   auth=(user, password))
    
    def query(self, query, parameters=None):
        with self.driver.session() as session:
            result = session.run(query, parameters or {})
            return [dict(record) for record in result]
  \end{lstlisting}
  
  \begin{block}{Utilisation}
    \texttt{conn = Neo4jConnection("bolt://localhost:7687", "neo4j", "password")}
  \end{block}
\end{frame}

\begin{frame}{Visualisation avec NetworkX}
  \begin{columns}
    \begin{column}{0.5\textwidth}
      \textbf{Fonctionnalités}
      \begin{itemize}
        \item Graphe du réseau d'amitiés
        \item Statistiques par étudiant
        \item Statistiques par cours
        \item Export en PNG haute résolution
      \end{itemize}
      
      \vspace{0.5cm}
      
      \textbf{Métriques calculées}
      \begin{itemize}
        \item Nombre d'amis
        \item Moyennes académiques
        \item Corrélations
        \item Répartition géographique
      \end{itemize}
    \end{column}
    
    \begin{column}{0.5\textwidth}
      \begin{center}
      \includegraphics[width=\textwidth]{images/reseau_amities.png}
      % Image générée par analyze.py
      \end{center}
    \end{column}
  \end{columns}
\end{frame}

\begin{frame}[fragile]{Utilisation : Guide pas à pas}
  \textbf{Étape 1 : Démarrer Neo4j}
  \begin{lstlisting}[language=bash]
docker-compose up -d
sleep 15  # Attendre le demarrage
  \end{lstlisting}
  
  \textbf{Étape 2 : Tester la connexion}
  \begin{lstlisting}[language=bash]
cd python/
python3 connect.py  # Doit afficher "Connecte !"
  \end{lstlisting}
  
  \textbf{Étape 3 : Peupler}
  \begin{lstlisting}[language=bash]
python3 populate.py
# Menu : Taper "3" puis "oui"
# Resultat : 6 etudiants, 5 cours, 16 amities
  \end{lstlisting}
\end{frame}

\begin{frame}[fragile]{Visualisation (suite)}
  \textbf{Étape 4 : Neo4j Browser}
  \begin{itemize}
    \item Ouvrir \url{http://localhost:7474}
    \item Login : neo4j / password123
    \item Requête : \texttt{MATCH (n) RETURN n LIMIT 50}
  \end{itemize}
  
  \vspace{0.3cm}
  
  \textbf{Étape 5 : Générer les graphiques}
  \begin{lstlisting}[language=bash]
python3 analyze.py
# Menu : Taper "5" (Tout generer)
# 3 images PNG creees dans ../images/
  \end{lstlisting}
  
  \vspace{0.3cm}
  
  \textbf{Étape 6 : Voir les résultats}
  \begin{lstlisting}[language=bash]
cd ../images/
xdg-open reseau_amities.png
xdg-open stats_etudiants.png
xdg-open stats_cours.png
  \end{lstlisting}
\end{frame}

\begin{frame}{Ce que vous visualisez}
  \begin{center}
  \begin{tikzpicture}[scale=0.8]
    % Neo4j Browser
    \node[draw, rounded corners, fill=blue!20, minimum width=3.5cm, minimum height=1cm] at (0,3) 
      {\small Neo4j Browser};
    \node[below, text width=3.5cm, align=center] at (0,2.3) 
      {\tiny Graphe interactif\\6 étudiants + 5 cours};
    
    % Python Graphiques
    \node[draw, rounded corners, fill=green!20, minimum width=3.5cm, minimum height=1cm] at (4.5,3) 
      {\small 3 Graphiques PNG};
    \node[below, text width=3.5cm, align=center] at (4.5,2.3) 
      {\tiny NetworkX + Stats};
    
    % UML
    \node[draw, rounded corners, fill=orange!20, minimum width=3.5cm, minimum height=1cm] at (0,1) 
      {\small Diagramme UML};
    \node[below, text width=3.5cm, align=center] at (0,0.3) 
      {\tiny Modèle de données};
    
    % PDFs
    \node[draw, rounded corners, fill=red!20, minimum width=3.5cm, minimum height=1cm] at (4.5,1) 
      {\small 2 Documents PDF};
    \node[below, text width=3.5cm, align=center] at (4.5,0.3) 
      {\tiny Rapport + Présentation};
  \end{tikzpicture}
  \end{center}
  
  \vspace{0.5cm}
  
  \begin{alertblock}{Temps total}
    Installation complète + visualisation : \textbf{~25 minutes}
  \end{alertblock}
\end{frame}

% Section 6
\section{Résultats}

\begin{frame}{Statistiques du réseau}
  \begin{center}
  \begin{tabular}{lc}
    \toprule
    \textbf{Métrique} & \textbf{Valeur} \\
    \midrule
    Étudiants & 6 \\
    Cours & 5 \\
    Amitiés & 16 (8 paires) \\
    Inscriptions & 17 \\
    Densité du réseau & 53\% \\
    Diamètre du graphe & 2 \\
    \bottomrule
  \end{tabular}
  \end{center}
  
  \vspace{0.5cm}
  
  \begin{exampleblock}{Interprétation}
    Réseau bien connecté avec un diamètre faible (tout le monde est à max 2 liens)
  \end{exampleblock}
\end{frame}

\begin{frame}{Top recommandations}
  \begin{center}
  \begin{tabular}{llc}
    \toprule
    \textbf{Étudiant} & \textbf{Suggestion} & \textbf{Amis communs} \\
    \midrule
    Thomas & Laura & 2 \\
    Sophie & Laura & 2 \\
    Rachidi & Laura & 1 \\
    Marie & Ahmed & 1 \\
    \bottomrule
  \end{tabular}
  \end{center}
  
  \vspace{0.5cm}
  
  \begin{alertblock}{Insight}
    Laura pourrait être introduite à Thomas et Sophie via leurs amis communs
  \end{alertblock}
\end{frame}

\begin{frame}{Types d'amitiés}
  \begin{center}
  \begin{tikzpicture}
    \pie[text=legend]{50/Amitiés proches (force $\geq$ 8),
                      37/Amitiés d'études,
                      13/Amitiés sportives/loisirs}
  \end{tikzpicture}
  \end{center}
  
  \vspace{0.5cm}
  
  \begin{block}{Observation}
    La moitié des amitiés sont considérées comme \textbf{proches} (force $\geq$ 8)
  \end{block}
\end{frame}

% Section 7
\section{Scénarios avancés}

\begin{frame}[fragile]{Détection de communautés}
  \begin{lstlisting}[language=Cypher]
MATCH path = (e1:Etudiant)-[:AMI_AVEC*2..3]-(e2:Etudiant)
WHERE e1 <> e2 AND 
      ALL(n IN nodes(path) WHERE 
          (n)-[:AMI_AVEC]-(e1) AND (n)-[:AMI_AVEC]-(e2))
RETURN DISTINCT [n IN nodes(path) | n.nom] as clique
LIMIT 5;
  \end{lstlisting}
  
  \vspace{0.3cm}
  
  \begin{block}{Cliques identifiées}
    \begin{itemize}
      \item \{Rachidi, Marie, Ahmed\} : groupe parisien
      \item \{Marie, Sophie, Thomas\} : groupe d'études
    \end{itemize}
  \end{block}
\end{frame}

\begin{frame}[fragile]{Plus court chemin}
  \begin{lstlisting}[language=Cypher]
MATCH path = shortestPath(
  (e1:Etudiant {nom: 'Diallo'})
   -[:AMI_AVEC*]-(e2:Etudiant {nom: 'Dubois'})
)
RETURN [n IN nodes(path) | n.prenom] as chemin,
       LENGTH(path) as longueur;
  \end{lstlisting}
  
  \vspace{0.5cm}
  
  \begin{center}
  \begin{tikzpicture}[node distance=2cm]
    \node[circle, draw, fill=neo4jblue!30] (r) {Rachidi};
    \node[circle, draw, fill=green!30, right of=r] (m) {Marie};
    \node[circle, draw, fill=orange!30, right of=m] (l) {Laura};
    
    \draw[->, thick] (r) -- node[above] {AMI} (m);
    \draw[->, thick] (m) -- node[above] {AMI} (l);
  \end{tikzpicture}
  \end{center}
  
  \textbf{Chemin} : Rachidi → Marie → Laura (longueur = 2)
\end{frame}

\begin{frame}[fragile]{Influence des cours}
  \begin{lstlisting}[language=Cypher]
MATCH (e1:Etudiant)-[:AMI_AVEC]->(e2:Etudiant),
      (e1)-[:ÉTUDIE]->(c:Cours)<-[:ÉTUDIE]-(e2)
WITH c.nom as cours, COUNT(*) as nb_liens_amicaux
RETURN cours, nb_liens_amicaux
ORDER BY nb_liens_amicaux DESC;
  \end{lstlisting}
  
  \vspace{0.3cm}
  
  \begin{block}{Résultat}
    Les cours "Bases de données" et "Algorithmes" favorisent le plus les amitiés
  \end{block}
  
  \begin{exampleblock}{Implication}
    Travaux de groupe $\rightarrow$ création de liens sociaux
  \end{exampleblock}
\end{frame}

% Section 8
\section{Conclusion}

\begin{frame}{Points clés}
  \begin{columns}
    \begin{column}{0.5\textwidth}
      \textbf{Réalisations}
      \begin{itemize}
        \item[$\checkmark$] Modèle de données complet
        \item[$\checkmark$] 50+ requêtes Cypher
        \item[$\checkmark$] Scripts Python d'analyse
        \item[$\checkmark$] Visualisations NetworkX
        \item[$\checkmark$] Recommandations pertinentes
      \end{itemize}
    \end{column}
    
    \begin{column}{0.5\textwidth}
      \textbf{Compétences acquises}
      \begin{itemize}
        \item Modélisation en graphes
        \item Langage Cypher
        \item Analyse de réseaux sociaux
        \item Intégration Python-Neo4j
        \item Algorithmes de graphes
      \end{itemize}
    \end{column}
  \end{columns}
  
  \vspace{1cm}
  
  \begin{alertblock}{Neo4j : Idéal pour...}
    Relations complexes, chemins, recommandations, détection de patterns
  \end{alertblock}
\end{frame}

\begin{frame}{Améliorations possibles}
  \textbf{Court terme}
  \begin{itemize}
    \item Ajouter plus d'étudiants et de relations
    \item Intégrer l'évolution temporelle du réseau
    \item Recommandations multi-critères (hobbies + cours + ville)
  \end{itemize}
  
  \vspace{0.5cm}
  
  \textbf{Long terme}
  \begin{itemize}
    \item Interface web interactive (Flask/Django + D3.js)
    \item Algorithmes avancés (PageRank, Louvain, Label Propagation)
    \item Machine Learning pour prédiction de liens
    \item API REST pour exposer les fonctionnalités
    \item Intégration avec systèmes universitaires existants
  \end{itemize}
\end{frame}

\begin{frame}{Applications réelles}
  \begin{center}
  \begin{tikzpicture}[scale=0.8]
    \node[draw, rounded corners, fill=blue!20, minimum width=4cm] at (0,3) {Réseaux sociaux (Facebook, LinkedIn)};
    \node[draw, rounded corners, fill=green!20, minimum width=4cm] at (0,2) {Recommandations (Netflix, Amazon)};
    \node[draw, rounded corners, fill=orange!20, minimum width=4cm] at (0,1) {Détection de fraude (banques)};
    \node[draw, rounded corners, fill=red!20, minimum width=4cm] at (0,0) {Knowledge graphs (Google)};
    \node[draw, rounded corners, fill=purple!20, minimum width=4cm] at (0,-1) {Réseaux de citations scientifiques};
  \end{tikzpicture}
  \end{center}
  
  \vspace{0.5cm}
  
  \begin{exampleblock}{Impact}
    Les graphes sont partout : transport, logistique, santé, cybersécurité...
  \end{exampleblock}
\end{frame}

\begin{frame}[standout]
  \Huge Merci !
  
  \vspace{1cm}
  
  \Large Questions ?
  
  \vspace{1cm}
  
  \normalsize
  \textbf{Projet disponible sur :}
  
  \texttt{/home/rachidi/Base\_de\_données/exposé\_neo4j/}
\end{frame}

\end{document}
